\documentclass[
    Ramasse,
    Correction
]{PrRJeval}

\NewDocumentCommand{\auteur}{}{M.~\bsc{Pochet}}
\NewDocumentCommand{\anneeScolaire}{}{2024/2025}
\NewDocumentCommand{\etablissement}{}{Lycée Condorcet (Lens)}
\NewDocumentCommand{\chapitre}{}{chapitre évalué}
\NewDocumentCommand{\noteMaximale}{}{20}

\begin{document}

{
    \itshape\noindent
    Le barème est donné à titre indicatif. La calculatrice est autorisée en \textbf{\textcolor{red}{mode examen}}. Sauf mention contraire, il est attendu que les réponses soient \textbf{\textcolor{red}{rédigées et justifiées}}. Toute trace de recherche, même incomplète, ou d'initiative même infructueuse, sera prise en compte dans l'évaluation. Le sujet est à rendre mais ne sera pas lu lors de la correction.\bigskip
}

\begin{cours}*[1]
    Quel est le titre du chapitre ?

    \begin{Correction}
        C'est le chapitre du moment.
    \end{Correction}
\end{cours}

\begin{exercice}*[2,5]
    Quelle est la couleur du cheval blanc d'Henri~\textsc{iv}?

    \begin{Correction}
        Il est blanc! 

        C'était pourtant évident, non?
    \end{Correction}
\end{exercice}

\begin{exercice}
    Le chiffre des dixièmes de $\pi$ est \correction{1}.
\end{exercice}

\newpage

Sujet plus long

\newpage

Sujet trop long ?

\end{document}